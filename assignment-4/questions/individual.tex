\documentclass[12pt]{article}
\usepackage{amsmath}
\usepackage{amsthm}
\usepackage{algorithm}
\usepackage{algpseudocode}
\usepackage{amssymb}
\usepackage{graphicx}
\graphicspath{ {assets/} }

\begin{document}

\title{Assignment 4 - Trapezoidal Map, Arrangements and Duality }
\author{
	Pattarawat Chormai - 0978675 \\
}
\maketitle

\section*{Q1}

As we know, a given simple polygon $P$ has $n$ vertices, $n$ edges. Thus, its trapezoidal
map $\tau$ has $3n+1$ trapezoids, or $O(n)$.

Regarding Vertical Decomposition definition, there is not line crossing a
trapezoid. Moreover, if one traverses from left to right and connect $leftp(\Delta_i)$
and $rightp(\Delta_i)$ together for each trapezoid $\Delta_i$, the approach will be similar
to $MakeMonotone$ algorithm that has a sweep line moving horizontally.


Thus, we can traverse all $\Delta \in \tau \cap P$ and connect $leftp(\Delta_i)$ and
$rightp(\Delta_i)$ together if possible. As a result, $P$ is divided into x-monotone
subpolygons. Hence, $TriangulateMonotonePolygon$, from the 3rd lecture Art Gallery problem,
can be used to triangulate the subpolygons.

% Figure: image x-monotone.

Note that, $TriangulateMonotonePolygon$ has to
be modified slightly to work with x-monotone polygon. \\


The algorithm works as follows:

\begin{enumerate}
    \item Traverse all $\Delta_i \in \tau \cap P$. If $leftp(\Delta_i)$
    and $rightp(\Delta_i)$ are not the same edge of $P$, connect them together
    as shown in Figure \ref{fig:x-monotone}.
    \item Apply $TriangulateMonotonePolygon$
\end{enumerate}

\begin{center}
    \label{figure1}
    \begin{figure}[h]
    \centering
    \includegraphics[width=5cm]{x-monotone}\\
    \caption{Connecting between $leftp(\Delta_i)$ and $rightp(\Delta_i)$ } \label{fig:x-monotone}
    \end{figure}
\end{center}

\subsection*{Correctness}
We know already that every edge connecting $leftp(\Delta_i)$ and $rightp(\Delta_i)$
is a valid splitting edge. Thus, one thing that remained to prove is
after step 1, the subpolygons are x-monotone.


According to monotone definition, a polygon will be monotone if there is no
split and merge vertex in the polygon. We know that only $leftp(\Delta_i)$ can be
a merge vertex and only $rightp(\Delta_i)$ can be a split vertex.

Considering $leftp(\Delta_i)$, if it is a merge vertex. After step 1, it will be
connected to $rightp(\Delta_i)$. Such vertex will split the polygon into 2 x-monotone
subpolygons. On the other hand, if $rightp(\Delta_i)$ if it is a split vertex.
It will be connected to $leftp(\Delta_{i})$ which also split polygon into 2 x-monotone
subpolygons. Hence, after step 1, $P$ will be divided into x-monotone subpolygons.

Because, in step 2, we use $TriangulateMonotonePolygon$ which has been proved correctness already.

Therefore, with this reasoning, we can conclude that the algorithm report correct result.

\subsection*{Running time}
We know that the number of trapezoids in $\tau$ is $O(n)$. Thus, traversal each trapezoid $\Delta$ that
are in $P$ and connect $leftp(\Delta_i)$ and $rightp(\Delta_i)$ takes $O(n)$. Because,
$TriangulateMonotonePolygon$ also takes $O(n)$. Therefore, the algorithm can compute
a triangulation of $P$ in $O(n)$.

\end{document}
