\documentclass[12pt]{article}
\usepackage{amsmath}
\usepackage{amsthm}
\usepackage{algorithm}
\usepackage{algpseudocode}
\usepackage{amssymb}
\usepackage{graphicx}
\graphicspath{ {assets/} }

\begin{document}

\title{Assignment 4 - Trapezoidal Map, Arrangements and Duality }
\author{
	Pattarawat Chormai - 0978675 \\
}
\maketitle

\section*{Q1}

As we know, a given simple polygon $P$ has $n$ vertices. Thus, its trapezoidal
map $\tau$ has $3n+1$ trapezoids, or $O(n)$.

Regarding Vertical Decomposition definition, there is not line crossing a
trapezoid. Moreover, if one traverses from left to right and connect $leftp(\Delta_i)$
and $rightp(\Delta_i)$ together for each trapezoid $\Delta_i$, the approach will be similar
to $MakeMonotone$ algorithm that has a sweep line moving horizontally.


Thus, we can traverse all $\Delta \in \tau \cap P$ and connect $leftp(\Delta_i)$ and
$rightp(\Delta_i)$ together if possible. As a result, $P$ is divided into x-monotone
subpolygons. Hence, $TriangulateMonotonePolygon$, from the 3rd lecture Art Gallery problem,
can be used to triangulate the subpolygons.

% Figure: image x-monotone.

Note that, $TriangulateMonotonePolygon$ has to
be modified slightly to work with x-monotone polygon. \\


The algorithm works as follows:

\begin{enumerate}
    \item Traverse all $\Delta_i \in \tau \cap P$. If $leftp(\Delta_i)$
    and $rightp(\Delta_i)$ are not the same edge of $P$, connect them together.
    \item Apply $TriangulateMonotonePolygon$
\end{enumerate}

% \subsection*{Correctness}
% We can retrieve \Delta in O(n)
% 1. Prove after connectin the dots, while we get x-monotone correctly
% 
% \subsection*{Running timek}
% - Number of \Delta is O(n) connect the dot take O(n)
% - Triangulatemonotonepolygon O(n)


\end{document}
