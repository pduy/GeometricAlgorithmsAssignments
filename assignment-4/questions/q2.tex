% !TEX root = ../main.tex

\section*{2}
\subsection*{a}
Search path to $q$ at $D_j$ becomes longer if $q$ is in a trapezoid that was just
created by the latest insertion. We also know that at most 4 line segments define
that trapezoids. Thus, the probability that the search part becomes longer is :
$$
Pr[\text{Search Path to $q$ becomes longer at step $i$}] = 4/i
$$

Hence, the expected increment of the length of the search part at step $k$, comparing to step $j$,
where $j<k$, is:

\begin{align*}
    \text{Expected Length} &\leq \sum_{i=j}^{k}( 4/i ) \\
    &= 4( \sum_{i=1}^{k}( 1/i ) -  \sum_{i=1}^{j}( 1/i ) ) \\
    &\leq 4( 1 + \ln k - 1 - \ln j ) \\
    &= O(\log(k/j))
\end{align*}

Therefore, the expected time locating $q$ at $D_k$ is $O(\log(k/j))$.

\subsection*{b}
We observe that vertical decomposition lines of $e_i$ intersects properly with $P$ if 
the edges that they intersect with have not been added into $T_i$ yet. Hence,
the number of proper intersections at $T_i$ is :

$$
\text{Number of Proper Intersections of $T_i$} = Pr[\text{$T_i$ has proper intersections}] = (n-j)/(n-1)
$$

Thus, the expected number of proper intersections, $E$, between $T_j$ and $P$ is:

\begin{align*}
    \text{E} &= \sum_{i=1}^{n}(n-i)/(n-1) \\
    &= \frac{1}{n-1}\frac{(n-1)n}{2} \\
    &= \frac{n}{2} \\
    &= O(n)
\end{align*}

\subsection*{c}
We analyse the expected running time of the algorihtm by dividing it into subprocesses.

For Line 4 - 5, the insertion takes $O(i/2^{(h-1)^2})$ as proven in $(a)$. Hence, in total
expected running time of the process takes, $R_{4\rightarrow5}$, is :
\begin{align*}
R_{4\rightarrow5} &= \sum_{i=2^{(h-1)^2}}^{2^{h^2}}O(i/2^{(h-1)^2}) \\
&= \frac{1}{2^{(h-1)^2}} ( \sum_{i=1}^{2^{h^2}}O(i) - \sum_{i=1}^{2^{(h-1)^2}}O(i) ) \\
&= \frac{1}{2^{(h-1)^2}} ( \frac{2^{h^2}(2^{h^2} + 1)}{2} - \frac{2^{(h-1)^2}(2^{(h-1)^2} - 1)}{2} ) \\
&= \frac{O(2^{2h^2})}{2^{(h-1)^2}} \\
&= O(2^{h^2})
\end{align*}

For Line 6, the running time of finding the leaf containing each vertex of $P$, is depend
on the number of proper intersections between $P$ and $T$. Thus, $R_{6}$ is:
\begin{align*}
    R_6 &= O(n)
\end{align*}

Hence $R_{3\rightarrow6}$ is:

\begin{align*}
    R_{3\rightarrow6} &= \sum_{h=1}^{\sqrt{\log{n}}}( O(2^{h^2}) + O(n) ) \\
    &= O(n\sqrt{\log{n}})
\end{align*}

For Line 8, because we start from $D_{n}$, hence, $R_{8}$ is $O(n)$ and we do Line 7 only
2 times, $n+1$ to $n$. Thus, $R_{7\rightarrow8}$ is $O(n)$.

Therefore, the expected running time of the algorithm is $O(n\sqrt{\log{n}})$.
