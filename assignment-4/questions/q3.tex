% !TEX root = ../main.tex
\section*{3}
\subsection*{a}
Let $s_{i^*}$ be the region in the dual plane corresponding to a line segment in $S$ and $l^*$ is the point
of $l$ in the dual plane.  Thus, the problem can be formulated as finding the number
of $s_{i^*} \in S^*$ that $p^*$ belongs to.

\subsection*{b}
Now the input turns out to be a set of $n$ pairs of intersected lines and a query point $p$.  We know that arrangement of the segments $A(S^*)$ has complexity $O(n^2)$,
$O(n^2)$ vertices and edges. Thus, we can use a DAG $D$ which is used in vertical decomposition as a query data structure.
Every trapezoid is stored in a leave of $D$, and stores the information about how many regions (corresponding to the original line segments) it belongs to.

Because we build $D$ from $A(S^*)$, the expected storage complexity of $D$ becomes $O(n^2)$.
Moreover, the depth of $D$ is $O(\log n^2)$, thus, the expected query time for searching a point is still $O(\log n)$.

\subsection*{c}

