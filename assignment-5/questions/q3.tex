% !TEX root = ../main.tex
\section*{3}
\subsection*{a}
In this section, we have to prove that the edge set of the EMST of $P$ contains the edges of a nearest neighbor graph.\\

Assume that there is an edge between the points $ a $ and $ b $ of the nearest neighbor graph which is not in the EMST, then we could replace the edge containing $a$ by $ab$ in the tree and that will decrease the total edge length of the EMST, because the current edge containing $a$ is longer than $ab$. So the EMST is not EMST. This contradiction proves the statement.

\subsection*{b}
In this section, we have to prove that the set of edges of the Gabiel graph of $P$ contains and EMST of $P$. \\

Assume that the edge set of the Gabiel graph does not contain an EMST of $P$, which means there is an edge $pq$ in the EMST that is not in the Gabiel graph. \\

This implies there is a point $k$ inside the circle taking $pq$ as the diameter. Hence,
$pk$ and $qk$ will become smaller than $pq$. Thus, $pq$ is not a valid edge of the EMST.
This contradiction proves the statement.

\subsection*{c}
Suppose that the Delaunay graph of $P$ does not contain the Gabriel graph of $P$, which means there are a pair of points ${p, q}$ that are not in the same triangle, but the circle $c$ taking $pq$ as the diameter is empty. We will prove that at least a triangle in the triangulation is illegal. \\

Indeed, there is several triangles between $p$ and $q$ because $pq$ is not a triangle edge. Let $q_1, q_2$ be 2 points in one of these triangles, which contains $p$.
We know that the circle $c$ is empty, so both $q_1$ and $q_2$ are outside of $c$. Thus, the 
circumcircle of $\bigtriangleup q_1 q q_2$ is bigger than $c$, which implies that its diameter is larger than $pq$. Then, this circumcircle contains $p$. So $\bigtriangleup q_1 q q_2$ is not a valid triangle. \\

Similarly, we can prove that the triangle containing $p$ is also invalid. Thus, the graph is not the Delaunay graph anymore. \\

By contradiction, we have just proven that the Delaunay graph of $P$ contains the Gabiel graph of $P$.

\subsection*{d}

Firstly, we have to prove that if $pq$ is an edge in the Gabriel graph, then the Delaunay edge between $p$ and $q$ intersects its dual Voronoi edge.\\

Assume $\bigtriangleup pvq$ a Delaunay triangle. Due to Gabriel graph definition, $v$ should not be in the region of the circle taking $pq$ as the diameter.\\

We know that the common intersection of the 3 Voronoi edges is the circumcenter of $\bigtriangleup pvq$. In this case, $\angle pvq \leq \pi / 2$ because v is outside of the circle, so the circumcenter is inside the triangle region. Thus, the Voronoi edge between $p$ and $q$ intersects $pq$.

Secondly, we have to prove that if the Delaunay edge between $p$ and $q$ intersects its dual Voronoi edge, then $pq$ is an edge in the Gabriel graph. \\

Denote $s$ the common intersection of bisectors $pq$, $qv$ and $pv$. $s$ is thus equidistant from $p, q, v$. \\

Because the Delaunay edge between $p$ and $q$ intersects its dual Voronoi edge, $s$ is in the half-plane defined by $pq$ and containing $v$. \\

We know that $sv = sq = sp = r$ where $r$ is the radius of the circle taking $pq$ as the diameter. Therefore, $v$ is not in the circle inner region.\\

This completes the proof.
