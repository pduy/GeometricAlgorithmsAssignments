% !TEX root = ../main.tex

\section*{1}

\begin{algorithm}[h]
  \caption{SmallConvexHull}
  \label{alg:smallconvexhull}
  \begin{algorithmic}
      \Require set of points $P$
      
      \If {$|P| < 5 $}
      	\State return true
      \EndIf

     \State Find $P_1$, $P_2$, the left-most and the right-most points from $P$
     \State Find $P_3 \in P$, which is the farthest point from the line $P_1 P_2$     
     \State Find $P_4 \in P$, which is the farthest point from the triangle $P_1 P_2 P_3$ and outside the triangle region.
     
     \If{$P4$ does not exist}
     \State return true
     \EndIf
     
     \If {One point in $P$ is outside the polygon $P_1 P_2 P_3 P_4$}
     \State return false
     \EndIf
       
     \State return true

\end{algorithmic}
\end{algorithm}

\begin{proof}
We will prove that the algorithm returns the correct result. \\

The convex hull covers all of the points in the set ($P$), by definition. Therefore, it covers the left-most and the right-most points; so $P_1$ and $P_2$ belong to the resulting convex hull of the set $P$. \\

$P_3$ is the farthest point from the line $P_1 P_2$. If $P_3$ does not belong to the convex hull, then the convex hull does not cover $P_3$. It contradicts the definition of the convex hull. Thus, $P_3$ must belong to the convex hull. \\

Similarly, $P4$ is the farthest point from the triangle $P_1 P_2 P_3$, which means $P4$ must belong to the convex hull. \\

If in the set $P$, there is a point outside of the polygon $P_1 P_2 P_3 P_4$, then we need more points to construct the convex hull because these 4 points are proven to be in the resulting convex hull. Thus, the convex hull contains more than 4 vertices. Otherwise, the convex hull obviously contains less than 5 vertices. \\

Therefore, the algorithm is correct. \\

Now, we will prove that the algorithm runs in $O(n)$ time. \\

For finding $P_i, 1 \leq i \leq 4$, it takes $O(n)$ time. \\

For checking that if any point is outside of the polygon, it takes $O(n)$ time. \\

Therefore, the overall time complexity is $O(n)$.
\end{proof}

