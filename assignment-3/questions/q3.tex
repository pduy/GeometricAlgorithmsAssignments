% !TEX root = ../main.tex
\section*{3}
\subsection*{a}

Because $p \in A$ and $q \in B$ where ${A,B}$ is a pair in the $s$-WSPD, the distance $|pq|$ should be large. In particular, $|pq| \geq sr$ where $r$ is the radius of $A$ and $B$. As we know, $s > 2$, so $sr > d$ where $d = 2s$ is the diameter of the circles. \\

If there is another point $p'$ in $A$, then $|pp'| \leq d < sr < |pq|$. Thus $q$ is not the nearest neighbor of $p$ anymore. This contradicts our proposal, which completes the proof.\\

If $p$ and $q$ are the closest pair in $P$, then $q$ is the nearest neighbor of $p$ and vice versa. Thus if $p \in A$ and $q \in B$ where ${A,B}$ is a pair in the $s$-WSPD, $A$ and $B$ contain only 1 element as being proven above. From the lecture, we know that the number of pairs in the $s$-WSPD is $O(s^d \dot n)$. $s$ and $d$ do not depend on $n$, so we can consider this as $O(n)$. Therefore, in $O(n)$ time, we can extract all pairs ${A_i, B_i}$ in which both groups contain only 1 element. Then we can compare the distances between the items in each pair, and pick the one with the smallest distance, which also takes $O(n)$ because the number of such pairs is at most the number of the $s$-WSPD. In general, we can do that in $O(n)$ time.\\


