% !TEX root = ../main.tex

\section*{3}
\subsection*{a}

This classification problem will be solved by a randomized incremental algorithm, by first shuffling the points randomly and initially choosing two points of different classes (P1 and P2). The line that pass through both  of them will be our initial separating line $l$. \\
\\
Note: that the separating line divides a plane into 2 halves. For a point to be in a proper position. that means it lies in the half plane which contains only points of the same class. \\
\\
Then, we will process the next points one by one. and for each point, we will continue looping if it's already in the proper position , if it's not. Our separating line will be changed to pass through this new point and the point of the different class in the previous line. then we will loop on every point that we have so far ( we will store every point that we already processed in an Array data structure Z) and check if every point in the proper position. 
\\
\\
\textbf{Algorithm:}\\
\\
$FindSeparatingLine$($S$):
\begin{enumerate}
\item Given: Set S of m points of class P1 and n point of class P2 
\item Output: Separating line ( it contains 2 points of different classes) if any or "not found"

\item S $\leftarrow$ RandomPermutation(S) 
\item Initialize $l$ our initial separating line to be the line between 2 different random points of P1 and P2
\item For every other point  $p  \in S$:
\begin{enumerate}
\item If $p$ in the proper position
\begin{enumerate}
\item continue 
\end{enumerate}
\item Else  
   \begin{enumerate}
\item $l$  $\leftarrow$ create new line between $p$ and the point of different class in $l$ 
\item loop on every point that we have so far and if there is a point that not in the right position return "not found"

\end{enumerate}    
        
\end{enumerate}

\item return $l$
\end{enumerate}

\subsubsection*{Correctness}
Indeed, the algorithm is correct. We will prove it by induction. \\ 

Based on our selection, the classifier $l$ always contains 1 point from $P1$ and 1 point from $P2$. \\

If the dataset contains only one point in $P1$ and one point in $P2$, then the algorithm pick the line $l$ passing these points to be the classifier, and this is the correct one. \\

Assume that after processing point $i$, the algorithm has already had the correct solution. Let $P*$ be the next point, and $P_i$ is the point of the same class ($P1$ or $P2$) which is 
currently on $l$.  When we insert $P*$, the new classifier should group it to the correct group. \\

If $P*$ is in the correct order already (the point of $P1$ should be above or on $l$, and the point of $P2$ should be below or on $l$), then the classifier $l$ is correct and
the algorithm does not do anything. If $P*$ is not in the correct order, which means it is on the ``outer'' space of the group of $P_i$, the line $l$ should be modified to group $P*$ to the proper position. Our algorithm decides to rotate the line $l$ so that it contains $P*$. This is a proper position for $P*$ because every point can be on $l$, and $P_i$ is also in that group because $l$ was rotated to the ``outer'' direction. Since the old $l$ containing $P_i$ was the correct classifier, this guarantees that the new line $l$ correctly classify the class of $P*$. \\ 

For the opposite class, the algorithm performs a check to see whether the new classifier $l$ is doing well for them or not. If all of the points are correctly positioned, then $l$ is obviously the correct classifier. If there are some points which are not in the right position, then there are no line classifiers available for this setting. This is because if we want to include those points, we have to move the line $l$ to the ``outer'' direction of this group, which means the ``inner'' direction of the class of $P*$. Such a movement will remove $P*$ from the line $l$, and $P*$ will be outside of the its proper group. Therefore, there are no agreements in this case. The algorithm indeed returns false, which is correct. \\

So, if the previous step is correct, then the current step is also correct. This implies that our incremental approach is correct. \\
